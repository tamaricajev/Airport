\documentclass{article}

\usepackage{float}
% \usepackage{floatrow}
\usepackage[T1, T2A]{fontenc}
\usepackage[english, serbian c]{babel}
\usepackage[utf8]{inputenc}
\usepackage{xcolor}

\usepackage[a4paper, total={6in, 8in}]{geometry}

\usepackage{amsmath}
\usepackage{graphicx}
\usepackage[colorlinks=true, allcolors=blue]{hyperref}

\usepackage{hyperref}
\hypersetup{colorlinks,linkcolor={black},citecolor={blue},urlcolor={blue}}  

\title{\textbf{ИНФОРМАЦИОНИ СИСТЕМИ} \\
\vspace{10}
\Large{\textbf{АЕРОДРОМ}}}
\author{\\\\Тамара Томић, 1046/2023 \\ \textit{mi231045@alas.matf.bg.ac.rs} \\\\
        Тамара Ђукић, 1051/2023 \\ \textit{mi231051@alas.matf.bg.ac.rs} \\\\
        Тамара Јевтимијевић, 1045/2023 \\ \textit{mi231045@alas.matf.bg.ac.rs} \\\\
        Милош Милаковић, 1052/2021 \\ \textit{mi211052@alas.matf.bg.ac.rs} \\\\\\
        \textit{професор:} др Саша Малков \\
        \textit{асистент:} Дара Милојковић \\\\\\}

\date{Београд 2023.}

\begin{document}

\maketitle
\thispagestyle{empty} 

\vspace{17}
\begin{figure}[h!]
    \centering
    \includegraphics[width=4cm, height=4cm]{grb.png}
\end{figure} 

\newpage
\tableofcontents

\newpage

\section{Увод}
Овај рад представља пројекат из предмета \textit{Информациони системи} на мастер студијама Математичког факултета, Универзитета у Београду. Рад описује информациони систем аеродрома. 
 Свесни смо да у данашњем времену, много људи на дневном нивоу користи аеродроме, па је у складу са тим потребно да сваки аеродром има систем који савршено ради у сваком моменту. На пример, што се тиче наше домаће авио-компаније, Air Serbia је у првих шест месеци обавила 63\% више летова и превезла 87\% више путника него за исти период 2022. године \cite{bbc_srbija}. \\ 
 Дакле, наша идеја је да развијемо систем, који ће се користити приликом рада аеродрома.\\
У оквиру анализе система препознати су основни процеси и учесници у систему. За израду дијаграма у оквиру пројекта коришћени су следећи алати:
\begin{itemize}
    \item Visual Paradigm
\end{itemize}

\section{Анализа система}

\subsection{Функционисање аеродрома}
Информациони систем, који развијамо, развијамо за потребе аеродрома. На аеродрому постоје две главне групе људи, а то су путници и радници. Како функционише аеродром? Путник дође на аеродром, чекира се, преда свој пртљаг и оде на гејт да чека укрцавање. То звучи једноставно, али иза свега тога постоји један дугачак процес. Да би путник уопште могао да лети, потребан је авион. Да би авион постојао потребно је да постоји авио-компанија, која поседује тај авион и која има уговор са аеродромом. Под уговором са аеродромом, подразумева се то да авио-компанија може да користи тј. да резервише писту аеродрома и да је користи за полетање и слетање својих авиона. Даље да би ти авиони били савршено исправни у сваком моменту потребно је да постоје људи који ће одржавати и контролисати те авионе. Затим да би авиони безбедно летели постоје људи који ће их пратити и усмеравати, итд. Као што видимо систем једног аеродрома је веома комплексан и компликован. На све то, у сваком моменту, мора савршено да функционише. \\
    На слици \ref{fig:uc_diagram} налази се дијаграм, који приказује учеснике система и њихове послове.

\subsection{Учесници у систему}
Основна подела учесника је на \textbf{путнике, запослене и авио-компаније}.\\
Запослени на аеродрому:
\begin{itemize}
    \item Администратор - има улогу да одржава базу података
        \begin{itemize}
            \item Администратор задужен за резервацију писте.
            \item Администратор задужен за уношење и брисање запослених.
            \item Администратор задужен за уношење и брисање авио-компанија.
        \end{itemize}
    \item Авио-механичар - задужен за поправку, проверу и одржавање авиона;
    \item Службеник - задужен за опслуживање путника;
    \item Контролор лета - задужен за праћење и навигирање авиона.
\end{itemize}

\subsection{Коришћени дијаграми и алати}
Током израде рада коришћени су следећи дијаграми:
\begin{itemize}
    \item UML дијаграми:
        \begin{enumerate}
            \item Дијаграм случајева употребе;
            \item Дијаграм активности;
            % \item Дијаграм секвенци.
        \end{enumerate}
    % \item BPMN дијаграми ($Business$ $Process$ $Modeling$ $Notation$ $Diagram$);
    % \item ER дијаграм ($Entity$ $Relation$ $Diagram$);
    % \item Дијаграми за приказ архитектуре система;
    % \item Скице корисночног интерфејса.
\end{itemize}

За израду свих UML дијаграма
%, као и ER и BPMN дијаграма 
коришћен је алат 
\textit{Visual Paradigm - Online/Community Edition} \cite{vs}.\\\\
% За израду скица корисночног интерфејса коришћен је алат $Diagrams.net$.

\begin{figure}[h!]
\centering
\includegraphics[width=0.9\textwidth, height=12cm]{Dijagrami_slike/uc_diagram.png}
\caption{\label{fig:uc_diagram}Дијаграм случајева употребе}
\end{figure}

\section{Процеси и случајеви употребе}

\subsection{Административни случајеви употребе}

\subsubsection{Уношење/брисање авио-компаније у систем}
% * <tamara.jev@gmail.com> 16:13:02 09 Nov 2023 UTC+0100:
% Милош Милаковић

\subsubsection{Уношење корисника у систем}

% * <tamara.jev@gmail.com> 16:12:14 09 Nov 2023 UTC+0100:
% Уношење радника у систем
\begin{itemize}
    \item \textbf{Кратак опис:} Администратор уноси информације о новом кориснику који који у моменту уноса не постоји у систему. Систем прихвата информације и валидира их и враћа потврду о успеху или неуспеху уноса.
    \item \textbf{Учесници:}
        \begin{itemize}
            \item \textit{Администратор}
        \end{itemize}
    \item \textbf{Предуслови:} Администратор има приступ систему и има информације о кориснику ког уноси.
    \item \textbf{Постуслови:} Корисник је успешно додат у систем.
    \item \textbf{Основни ток:}
        \begin{enumerate}
            \item Администратор приступа систему и започиње поступак уноса корисника у систем тако што отвори формулар.
            \item Систем приказује форму за унос података.
            \item Администратор уноси све потребне информације у формулар.
            \item Администратор потврђује унос.
            \item Систем валидира унете податке.
            \item Систем уноси податке у базу података.
            \item Систем обавештава администратора да је операција успешно извршена.
        \end{enumerate}
    
    \item \textbf{Алтернативни токови:}
    \item А1. \textbf{Администратор одустаје од уноса података.} Ако у кораку 4 администратор кликне на дугме одустани подаци се не шаљу систему и неће бити унети у исти.
    \item A2. \textbf{Унети подаци о авио компанији нису исправни.} Ако у кораку 5 систем препозна да су неки од унетих података невалидни, избациће грешку и обавестити администратора које поље је невалидно. Администратор исправља грешку и наставља даље од 4. корака главног тока. 
\end{itemize}

\begin{figure}[H]
    \centering
    \includegraphics[width=1.1\textwidth, height=15cm]{Dijagrami_slike/dodavanje_korisnika.jpg}
    \caption{Дијаграм активности - Додавање корисника у систем}
\end{figure}

\subsubsection{Брисање корисника из система}
% * <tamara.jev@gmail.com> 16:12:29 09 Nov 2023 UTC+0100:
% Брисање радника из система

\begin{itemize}
    \item \textbf{Кратак опис:} Администратор брише корисника који постоји у систему. Систем ажурира базу и враћа информацију о успешности захтева.
    \item \textbf{Учесници:}
        \begin{itemize}
            \item \textit{Администратор}
        \end{itemize}
    \item \textbf{Предуслови:} Администратор има приступ систему и има информације о кориснику ког жели да обрише.
    \item \textbf{Постуслови:} Корисник је успешно обрисан из система.
    \item \textbf{Основни ток:}
        \begin{enumerate}
            \item Администратор приступа систему и отвара страну за управљање корисницима.
            \item Администратор уноси вредност за критеријум претраге.
            \item Администратор кликом на дугме врши претрагу.
            \item Систем валидира унос.
            \item Систем на основу вредности за критеријум претраге филтрира кориснике и враћа резултате администратору.
            \item На основу добијених корисника, администратор бира одговарајућег и подноси захтев систему за брисање.
            \item Систем брише корисника и ажурира базу.
            \item Систем обавештава администратора да је операција успешно извршена.
        \end{enumerate}
    
    \item \textbf{Алтернативни токови:}
        \begin{itemize}
            \item[А1.] \textbf{Администратор одустаје од уноса података.} Ако у кораку 3 или 6 администратор одустане од брисања корисника, захтев се не шаље систему и корисник неће бити обрисан.
            \item[A2.] \textbf{Унети подаци за филтрирање нису исправни.} Ако у кораку 4 систем препозна да је вредност за филтрирање невалидна, избациће грешку и обавестити администратора о грешци. Администратор исправља грешку и наставља даље од 3. корака главног тока. 
        \end{itemize}
\end{itemize}

\begin{figure}[H]
    \centering
    \includegraphics[width=1.1\textwidth, height=18cm]{Dijagrami_slike/brisanje_korisnika.jpg}
    \caption{Дијаграм активности - Брисање корисника из система}
\end{figure}

\subsection{Резервација аеродрома (писте):}

\begin{figure}[H]
    \centering
    \includegraphics[width=0.9\textwidth]{Dijagrami_slike/ucs_rezervacija_aerodroma.png}
    \caption{Дијаграм случаја употребе - Резервација аеродрома (писте)}
\end{figure}

\begin{itemize}
    \item \textbf{Кратак опис:} Авио-компанија контактира са администратором аеродрома задуженим за резервације писте у циљу резервисања писте за полетање или слетање својих авиона. Са администратором може да се договара око датума, времена као и цене резервисања писте.
    \item \textbf{Учесници:}
        \begin{itemize}
            \item \textit{Авио-компанија;}
            \item \textit{Администратор који је задужен за резервације писте.}
        \end{itemize}
    \item \textbf{Предуслови:} Авио-компанија поседује барем један авион и регистрована је тј. да има уговор о пословању са аеродромом.
    \item \textbf{Постуслови:} Авио-компанија или јесте или није резервисала писту.
    \newpage
    \item \textbf{Основни ток:}
        \begin{enumerate}
            \item Авио-компанија пише захтев за резервацију аеродрома у ком наводи тачан датум кад жели да резервише писту.
            \item Када је написала захтев, шаље га администратору аеродрома, који је задужен за резервације писте.
            \item Администратор проверава доступност писте за тражени датум и проверава која времена су доступна тог датума. 
                \begin{enumerate}
                    \item Ако не постоји ниједно време у које је писта доступна тог дана:
                        \begin{itemize}
                            \item Авио-компанија отказује резервацију.
                            \item Авио-компанија мења датум и то шаље администратору.
                        \end{itemize}
                    \item Ако постоји барем једно време у које је писта слободна, администратор шаље авио-компанији цену и време могуће резервације писте.
                \end{enumerate}
            \item Авио-компанија разматра да ли цена и време одговарају или не.
                 \begin{enumerate}
                    \item Ако не одговарају, авио-компанија разматра да ли жели да откаже резервацију или не. 
                        \begin{itemize}
                            \item Авио-компанија отказује резервацију.
                            \item Авио-компанија мења датум и то шаље администратору.
                        \end{itemize}
                    \item Ако цена и време одговарају, администратор врши резервацију писте.
                \end{enumerate}
        \item Комуникације између администратора и авио-компаније је завршена.
    \end{enumerate}
    \item \textbf{Алтернативни токови:}
        \begin{itemize}
            \item[A1.] \textbf{Унос неисправног датума.} Уколико у кораку 2 основног тока, администратор утврди да је послат погрешан датум, авио-компанија о томе бива обавештена и процес се наставља од корака 1 основног тока.
        \end{itemize}
\end{itemize}

\begin{figure}[H]
    \centering
    \includegraphics[width=1.1\textwidth, height=7cm]{Dijagrami_slike/rezervacija_aerodroma.png}
    \caption{Дијаграм активности - Резервација аеродрома (писте)}
\end{figure}

\subsection{Праћење летова}

\subsection{Обрада захтева}
% * <tamara.jev@gmail.com> 01:33:43 07 Nov 2023 UTC+0100:
% Tamara Djukic

\subsubsection{Чекирање путника}
% * <tamara.jev@gmail.com> 01:33:48 07 Nov 2023 UTC+0100:
% Tamara Djukic

%dijagram aktivnosti

\begin{itemize}
    \item \textbf{Кратак опис:} Путник долази на шалтер аеродрома са купљеном картом, пасошем и пртљагом. Службеник проверава исправност наведених докумената, и при успешној радњи, корисник добија карту за лет и иде даље на пасошку контролу.
    \item \textbf{Учесници:}
        \begin{itemize}
            \item \textit{Путник}
            \item \textit{Службеник}
        \end{itemize}
    \item \textbf{Предуслови:} Путник је купио карту преко интернета и са пасошом и пртљагом дошао на аеродром.
    \item \textbf{Постуслови:} Путник се успешно чекирао на лет и може да настави даље ка пасошкој контроли.
    \item \textbf{Основни ток:}
        \begin{enumerate}
            \item Путник на интернету купује карту за одређени лет.
            \item Путник долази на аеродром и приступа на шалтер код службеника.
            \item Путник показује службенику доказ о куповини карте.
            \item Службеник проверава да ли се карта налази у бази података авио компаније.
            \item Ако се карта налази у бази података, службеник даље проверава пасош и пртљаг.
            \item Ако је све исправно прошло, путник се успешно чекирао на конкретан лет.
            \item Путник наставља даље на пасошку контролу.
        \end{enumerate}
    
    \item \textbf{Алтернативни токови:}
    \item А1. \textbf{Путник одустаје од лета:} Ако у кораку 2. и 3. путник промени мишљење за летење авионом, у кораку 5. службеник не проверава постојање карте у бази података и ту се основни ток завршава.
\end{itemize}

\subsubsection{Чекирање запослених}
% * <tamara.jev@gmail.com> 01:33:52 07 Nov 2023 UTC+0100:
% Tamara Djukic

%dijagram aktivnosti

\begin{itemize}
    \item \textbf{Кратак опис:} Запослен долази на шалтер аеродрома са пасошом и опционим пртљагом. Службеник проверава исправност наведених докумената, и при успешној радњи, запослен иде даље на пасошку контролу.
    \item \textbf{Учесници:}
        \begin{itemize}
            \item \textit{Запослен}
            \item \textit{Службеник}
        \end{itemize}
    \item \textbf{Предуслови:} Запослен је са пасошом и опционим пртљагом дошао на аеродром.
    \item \textbf{Постуслови:} Запослен се успешно чекирао на конкретан лет и може да настави даље ка пасошкој контроли.
    \item \textbf{Основни ток:}
        \begin{enumerate}
            \item Запослен долази на аеродром и приступа на шалтер код службеника.
            \item Запослен показује службенику пасош.
            \item Службеник, на основу пасоша, проверава да ли се запослен налази у бази података авио компаније.
            \item Ако се налази у бази података, службеник даље проверава да ли је запослен распоређен за конкретан лет.
            \item Ако јесте рапоређен, службеник га убацује у систем да је дошао на лет.
            \item Ако је све исправно прошло, запослен се успешно чекирао на конкретан лет.
            \item Запослен наставља даље на пасошку контролу.
        \end{enumerate}
    
    \item \textbf{Алтернативни токови:}
    \item А1. \textbf{Запослен није распоређен за конкретан лет:} Ако у кораку 4. запослен није распоређен на конкретан лет, запослен одлази са аеродрома и ту се основни ток завршава.
\end{itemize}

\subsection{Одржавање авиона}
% * <tamara.jev@gmail.com> 01:32:03 07 Nov 2023 UTC+0100:
% Tamara Tomic

\subsubsection{Пријава квара}
% * <tamara.jev@gmail.com> 01:32:25 07 Nov 2023 UTC+0100:
% Tamara Tomic

\subsubsection{Поправка квара}
% * <tamara.jev@gmail.com> 01:33:18 07 Nov 2023 UTC+0100:
% Tamara Tomic

\subsubsection{Уношење потребних делова у систем}
% * <tamara.jev@gmail.com> 01:33:24 07 Nov 2023 UTC+0100:
% Tamara Tomic

\section{База података}

\section{Софтверска архитектура}

\section{Кориснички интерфејс}

\newpage
\bibliographystyle{unsrt}
\bibliography{bibliography} 


\end{document}